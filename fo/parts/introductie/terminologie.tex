\subsection{Basiselementen, terminologie en definities}
De technische realisatie wordt uitgevoerd op basis van het CMS framework Drupal, 
versie 7. Elk softwaresysteem hanteert bepaald taalgebruik en termen. Om gebruik 
te kunnen maken van de gangbare termen, niet telkens uitleg te herhalen en om 
verwarring te voorkomen worden hieronder een aantal termen eenmalig toegelicht. 

\begin{description}
\item[Drupal] Drupal is het gekozen Content Management Systeem wat als framework 
dient voor de bouw van de website. Drupal is een Open Source CMS dat geschreven 
is in de programmeertaal PHP. 
\item[Drupal.org] De website van Drupal. Dit is de thuisbasis van de community, 
waar Drupal zelf en de zogenaamde contrib modules gedownload kunnen worden. 
\item[Module] Drupal is een modulair systeem. Een module is een verzameling 
programmacode die de functionaliteit van het systeem uitbreidt of aanpast. In 
andere systemen wordt dit ook wel een plug-in of extensie genoemd. 
\item[Contrib (contributed) module] Wanneer gesproken wordt over contrib of 
contributed modules, worden modules bedoeld die door leden van de 
Drupal-community beschikbaar zijn gesteld op Drupal.org. Een Drupal-website 
gebruikt over het algemeen enkele tientallen contrib modules, varri\"erend van 
grote brokken functionaliteit tot kleine aanpassingen in de standaard 
functionaliteit van Drupal zelf of van andere modules.
\item[Custom modules] Zoals er contributed modules zijn die beschikbaar zijn op 
Drupal.org, is het voor een sitebouwer ook mogelijk om zelf een module te 
schrijven, omdat de functionaliteit niet beschikbaar is binnen de bestaande 
contrib modules.
\item[Contenttype] Elk content item in een Drupal site is van een bepaald type, 
met een zelfde structuur van velden. Hieraan kan bijvoorbeeld een bepaalde 
pagina-layout gekoppeld zijn, of items van een bepaald type (bijvoorbeeld het 
type ``Nieuwsartikel") worden weergegeven in een lijstweergave.
\item[Taxonomie] Voor de categorisering van inhoud wordt gebruik gemaakt van 
taxonomie. Hierbij kan sprake zijn van ``vaste" categori\"en (die wel in een 
apart scherm te beheren zijn, maar niet vrij toe te voegen bij het aanmaken van 
content), of van ``vrije" tags, waarbij een redacteur zelf nieuwe termen kan 
toevoegen door ze in te voeren bij het invoeren en bewerken van content. 
\item[Gebruiker] Een persoon die gebruik maakt van het systeem. Soms wordt 
hiermee ook de gebruikers\emph{account} bedoeld, de gegevens die van een 
gebruiker zijn opgeslagen in het systeem.
\item[Eindgebruiker] Een bezoeker van de website (dus niet een redacteur of 
administrator die beheerstaken uitvoert), al dan niet ingelogd. Dit onderscheid 
wordt binnen Drupal zelf niet gemaakt, het verschil tussen een gebruiker en een 
eindgebruiker zit impliciet in de toegekende \emph{rollen}. Het onderscheid kan 
bij functionele beschrijvingen wel nuttig zijn. 
\item[Rol] Elke gebruiker heeft minimaal \'e\'en rol binnen Drupal. Aan 
een rol zijn bepaalde toegangsrechten gekoppeld, bijvoorbeeld \emph{Kan reacties 
plaatsen} of \emph{Kan nieuwe content aanmaken}. Er zijn drie rollen standaard 
aanwezig in Drupal; \emph{anonieme gebruiker}, \emph{geverifieerde gebruiker} en 
\emph{administrator}. De sitebouwer kan zonodig rollen toevoegen, meestal wordt 
er minimaal nog een rol \emph{Redacteur} toegevoegd.
\item[Anonieme gebruiker] Een rol die een gebruiker automatisch heeft als hij 
niet is ingelogd.
\item[Geverifieerde gebruiker] Een rol die een gebruiker automatisch heeft als 
hij wel is ingelogd.
\item[Administrator] Een rol die automatisch \emph{alle} beschikbare rechten 
krijgt. Gebruikers die deze rol krij\-gen toegekend mogen alle pagina's en 
inhoud bezoeken en bewerken en mogen alle instellingen van een website 
aanpassen. 
\item[Blok] Drupal maakt veel gebruik van blokken (blocks). Een blok is een 
logisch samenhangend deel van een pagina, bijvoorbeeld een menu, een lijstje met 
de laatste nieuwsberichten in de rechter zijbalk, of de hoofdinhoud. Blokken 
kunnen aangeboden worden door modules waarmee de inhoud dynamisch bepaald kan 
worden, of vrij samengesteld worden door direct de (statische) inhoud in te 
voeren.
\item[Views] Een veelgebruikte contrib module, welke wordt gebruikt om 
overzichten en lijsten mee te maken. Als er gesproken wordt over ``een 
View" dan wordt er verwezen naar een lijst geimplementeerd met de Views-module. 
\item[Theme] Een verzameling van bestanden (templates, styling en 
afbeeldingen), die samen het uiterlijk van de website bepalen. Een theme bevat 
elementen zoals een header met o.a. het logo van de site, ikonen en plaatsing 
van \emph{regio's}.
\item[Regio] Een deel van een pagina waarin blokken geplaatst kunnen worden. 
\item[Template] Een bestand dat onderdeel uitmaakt van een theme en bestaat uit 
HTML met PHP code, bedoeld om de HTML-structuur van een pagina of een deel van 
een pagina vast te leggen. Bijvoorbeeld, een pagina-template bepaalt de 
plaatsing van het site-logo en van de regio's, en een template voor een 
contenttype bepaalt de plaatsing van velden.
\item[Cache] Drupal genereert (onderdelen van) pagina's en slaat deze op in de 
cache. Ook worden resultaten van relatief tijdrovende bewerkingen in de cache 
opgeslagen, zodat ze voor hergebruik snel beschikbaar zijn. Door dit systeem 
hoeft een pagina niet telkens volledig opnieuw opgebouwd te worden, wat de 
snelheid waarmee de pagina op het scherm verschijnt ten goede komt. 
\item[Core] Het basissysteem en basismodules van Drupal. Dat wat er in de 
download zit als alleen ``Drupal" wordt gedownload van Drupal.org. 
\item[Core hack] Hiermee wordt een aanpassing aan de code\footnote{Dit is dus 
iets anders dan een module; deze maakt gebruik van de beschikbare 
programmeer-interfaces om de functionaliteit uit te breiden of te veranderen. 
Hiervoor zijn geen aanpassingen aan de code van het standaard systeem nodig.} 
van het standaard systeem bedoelt (in principe mogelijk omdat de code Open 
Source is en dus vrij toegankelijk en aanpasbaar). Core hacks worden gezien als 
uitermate negatief, omdat het problemen oplevert bij het updaten naar nieuwe 
versies van de core en de functionaliteit op onverwachte manieren kan 
veranderen. Hetzelfde principe gaat op voor contrib modules, ook hier wordt 
vaak gesproken van een core hack, in dat geval dus niet de core van Drupal, maar 
de ``core'' van de contributed module.
\item[Cron] Een mechanisme binnen Drupal waarmee regelmatig terugkerende taken 
uitgevoerd kunnen worden. Te denken valt bijvoorbeeld aan het opschonen van 
bepaalde database-tabellen, of het legen van de cache. Een module kan door 
middel van een programmeer-interface dergelijke taken aanmelden.
\item[Entity] Een entity is een samenhangende verzameling informatie. 
Bijvoorbeeld, elk content item is een entity, net als een gebruikersaccount of 
een reactie.
\item[Veld] Een veld is een stuk informatie dat aan een entity gekoppeld kan 
worden. Te denken valt aan de titel van een artikel, de datum van een 
agenda-item of de \emph{avatar} (gebruikersafbeelding) van een 
gebruikersaccount. 
\item[Menu] Een verzameling navigatie-links (menu items). 
\item[Node] Een enkel content item wordt binnen de techniek van Drupal een 
\emph{node} genoemd. Deze term is volledig uitwisselbaar met de term 
\emph{content item}. 
\end{description}
