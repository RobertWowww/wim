\section{Rechten en rollen}

Drupal heeft de mogelijkheid om meerdere rollen aan te maken. Aan elke rol kunnen vervolgens rechten worden toegekend. Welke rechten er zijn is afhankelijk van de gebruikte modules een aanwezige contenttypes. Een gebruiker kan aan meerdere rollen worden gekoppeld. Doorgaans wordt \'{e}\'{e}n rol gebruikt voor redacteuren / beheerders.

\subsection{Rollen}

De volgende rollen zijn gedefin\"{i}eerd:
\begin{itemize}
\item Beheerder \\
	Heeft toegang tot beheerpagina's en kan alle inhoud, subsites en gebruikers beheren.
\item Eindredacteur \\
	Kan alle inhoud en team subsites beheren.
\item Redacteur \\
	Kan inhoud beheren op eigen domeinen. Kan wel inhoud aanmaken, maar niet publiceren.
\item Medewerker \\
	Toegang tot bepaalde functionaliteit die exclusief is voor medewerkers (bijv. wiki).
\item Teamlid \\
	Kan inhoud beheren binnen eigen team subsites.
\end{itemize}

\subsection{Inhoudstypen}

Hier worden de rechten op inhoud beschreven. Het aanmaken en publiceren zijn losse rechten. Wel kunnen aanmaken en niet kunnen publiceren betekend dat wijzigingen door een andere rol goedgekeurd moeten worden\seeone{workflow}. Bij het bewerken en verwijderen is er onderscheid tussen het muteren van eigen inhoud of het muteren van alle inhoud. Bij elke permissie zijn de rollen afgekort tot de eerste letter.
\begin{itemize}
\item \textbf{B}eheerder
\item \textbf{E}indredacteur
\item \textbf{R}edacteur
\item \textbf{M}edewerker
\item \textbf{T}eamlid
\end{itemize}
\begin{tabularx}{\textwidth}{ | p{5cm} |X|X|X|X|X|X| }
  \hline
  \textbf{Inhoudstype} & \rotatebox[origin=l]{90}{\textbf{Aanmaken}} & \rotatebox[origin=l]{90}{\textbf{Bewerken}} & \rotatebox[origin=l]{90}{\textbf{Alles bewerken}} & \rotatebox[origin=l]{90}{\textbf{Verwijderen}} & \rotatebox[origin=l]{90}{\textbf{Alles verwijderen }} & \rotatebox[origin=l]{90}{\textbf{Publiceren}} \\ \hline
  Agenda & BERT & BERT & BE & BERT & BE & BE \\ \hline
  Bekendmaking & B & B & B & B & B & B \\ \hline
  Intranet bericht & BET & BET & BE & BET & BE & BE \\ \hline
  Bestand & BET & BET & BE & BET & BE & BE \\ \hline
  Bestemmingsplan & BE & BE & BE & BE & BE & BE \\ \hline
  Blog & BERT & BERT & BE & BERT & BE & BE \\ \hline
  Editorial & BER & BER & BE & BER & BE & BE \\ \hline
  Eenvoudige pagina & BER & BER & BE & BER & BE & BE \\ \hline
  FAQ & BER & BER & BE & BER & BE & BE \\ \hline
  Forumonderwerp & BE & BE & BE & BE & BE & BE \\ \hline
  Foto & BE & BE & BE & BE & BE & BE \\ \hline
  Forumonderwerp & BE & BE & BE & BE & BE & BE \\ \hline
  Marktplaats & BERM & BERM & BE & BERM & BE & BE \\ \hline
  Nieuws & BER & BER & BE & BER & BE & BE \\ \hline
  Onderwerp & BE & BE & BE & BE & BE & BE \\ \hline
  Peiling & BER & BER & BE & BER & BE & BE \\ \hline
  Persoon & BER & BER & BE & BER & BE & BE \\ \hline
  Product & BER & BER & BE & BER & BE & BE \\ \hline
  Regeilng & B & B & B & B & B & B \\ \hline
  RSS & BE & BE & BE & BE & BE & BE \\ \hline
  RSS Source & BE & BE & BE & BE & BE & BE \\ \hline
  Slide & BER & BER & BE & BER & BE & BE \\ \hline
  VAC & BE & BE & BE & BE & BE & BE \\ \hline
  Webformulier & BE & BE & BE & BE & BE & BE \\ \hline
  Wiki & BERMT & BERMT & BE & BERMT & BE & BE \\ \hline
\end{tabularx}

\clearpage
\subsection{Permissies}

Het volgende overzicht geeft aan welke rollen toegang hebben tot bepaald acties.
Bij elke permissie zijn de rollen afgekort tot de eerste letter.

\begin{tabularx}{\textwidth}{ | p{10cm} | X | } \hline 
\textbf{Actie} & \textbf{Rollen} \\ \hline
Beheerinterface gebruiken 	& BER \\ \hline
Content uitsluiten van zoekmachine 	& BE \\ \hline
Andere invoerformaten kiezen 	& BERT \\ \hline
Reageren op content 	& BERTM \\ \hline
Eigen reacties bewerken 	& B \\ \hline
Blokken en context beheren 	& BE \\ \hline
Contextual links gebruiken 	& BER \\ \hline
Team subsites beheren 	& BE \\ \hline
Subsite menu's beheren 	& BER \\ \hline
Cron settings aanpassen 	& B \\ \hline
Felix blokken beheren op eigen domein 	& BER \\ \hline
Felix blokken beheren op alle domeinen 	& BE \\ \hline
Toegang tot alle uploads 	& B \\ \hline
Bestanden uploaden 	& BERT \\ \hline
Eigen bestanden aanpassen 	& BERT \\ \hline
Alle bestanden aanpassen 	& BE \\ \hline
Volledige HTML gebruiken 	& BE \\ \hline
Gefilterde HTML gebruiken 	& BERTM \\ \hline
Forums beheren 	& B \\ \hline
Vertalingen aanpassen 	& BE \\ \hline
MailChimp campagnes beheren 	& B \\ \hline
Inloggen als andere gebruikers 	& B \\ \hline
Externe video's toevoegen (via media) 	& BER \\ \hline
Nodequeues aanpassen 	& BER \\ \hline
URL Aliasen beheren 	& BE \\ \hline
Stemmen van de poll bekijken 	& BER \\ \hline
Publiceren onder embargo 	& BER \\ \hline
Termen in woordenlijsten beheren 	& BE \\ \hline
Gebruikers beheren 	& B \\ \hline
Gebruikersprofielen bekijken 	& BE \\ \hline
Inzendingen van formulieren bekijken 	& BE \\ \hline
Workbench gebruiken 	& BER \\ \hline
\end{tabularx}


