\subsection{Rollen}\label{rollen}

Rollen zijn een synoniem voor 'gebruikers typen', bijvoorbeeld: 'Anonieme gebruiker' en 'Administrator'.  Het is mogelijk om rollen specifieke rechten te geven. Op deze manier kun je bijvoorbeeld bepaalde gebieden of functionaliteiten van de website verbergen of juist tonen. 

\subsubsection{Administrator}\label{administrator}
Gebruikers met de rol 'Administrator' hebben ongelimiteerd toegang tot de gehele website. De 'Administrator' kan alle content bekijken, rechten toekennen en bijvoorbeeld menu's aanmaken en bewerken.

\subsubsection{Medewerker}\label{medewerker}
Gebruikers met de rol 'Medewerker' hebben een gelimiteerde toegang tot de website. Medewerkers kunnen o.a. content publiceren, nieuwe content toevoegen op de interne marktplaats en nieuwe wiki pagina's aanmaken.

\subsubsection{Redacteur}\label{redacteur}
Gebruikers met de rol 'Redacteur' hebben een gelimiteerde toegang tot de website. Redacteuren hebben o.a. toegang tot het beheermenu, mogen contextuele links gebruiken en kunnen inhoud aanmaken van alle inhoudstypen. Redacteuren moet geen content publiceren.

\subsubsection{Eindredacteur}\label{eindredacteur}
Gebruikers met de rol 'Eindredacteur' hebben een ongelimiteerde toegang op het gebied van content. Eindredacteuren mogen content aanmaken, wijzigen, publiceren en verwijderen. Ook hebben eindredacteuren toegang tot het beheermenu en mogen zij de contextuele links gebruiken.

\subsubsection{Teamlid}\label{teamlid}
De rol 'Teamlid' is bedoeld voor subsites. Deze rol kan aan een bestaand persoon toegekend worden om haar toegang te geven tot content op subsites en om haar toestemming te geven om content aan te maken op de opgegeven subsite(s).