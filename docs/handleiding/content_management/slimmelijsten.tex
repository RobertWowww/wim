\subsection{Slimme lijsten}\label{slimmelijsten}

Via de slimme lijsten module kan de eindredactie zelf lijsten aanmaken. Deze lijsten kunnen worden aangeboden als pagina, blok of beide.
Ga naar \emph{Structuur} $\Rightarrow$ \emph{Lijsten} voor een overzicht van de reeds bestaande lijsten, of ga direct naar \drupalpath{admin/structure/lists}.
Lijsten zijn gekoppeld aan de subsites. Alleen de lijsten van de huidige subsite worden getoond. Lijsten die worden aangemaakt zijn tevens enkel actief op de huidige subsite en worden dus niet gedeeld door bijv. internet en intranet.

\subsubsection{Nieuwe lijst aanmaken}

Klik in het overzicht\seeone{slimmelijsten} op \emph{Lijst toevoegen} om een nieuwe lijst aan te maken. Als eerste moet een titel en een type worden opgegeven. De titel wordt boven de lijst getoond. Het type is \emph{inhoud} wanneer de lijst nodes bevat en \emph{menu} wanneer onderliggende items van een specifiek menu-item worden getoond. De instellingen direct onder deze opties zijn afhankelijk van het gekozen type.

Wanneer gekozen is voor type \emph{Inhoud} dan heeft men de mogelijkheid om een nodetype te kiezen (bijv. \emph{agenda}). Er is ook een optie \emph{Alle} die alle nodetypes zal laten zien.
In het veld \emph{Term} kan een tag worden ingevoerd. Dit veld is optioneel. Indien dit wordt gebruikt zullen alleen resultaten worden getoond met deze tag. Hiermee kan bijvoorbeeld een lijst van gerelateerde items worden gemaakt. Tevens kan de sortering worden aangegeven.

Met de optie \emph{Blok aanbieden} kan worden gekozen of een blok beschikbaar moet zijn. De enige setting hiervoor is het aantal items dat wordt getoond in het blok. Indien er ook een pagina beschikbaar is zal een meer-link worden getoond in het blok. De blokken zelf kunnen na aanmaken van de lijst op een pagina worden toegevoegd via felix\seeone{felix}.

De optie \emph{Pagina aanbieden} wordt gebruikt om een pagina aan te maken. Hiervoor moet een pad worden opgegeven (bijv. "projecten" voor \texttt{gemeente.nl/projecten}). Tevens wordt het aantal items opgegeven (standaard 10). Overige items zijn beschikbaar via paginanummering. Als laatste is het mogelijk om een introductietekst in te voeren.