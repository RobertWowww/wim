\clearpage
\subsection{Workflow}\label{workflow}

Onderstaand tabel toont de workflow per inhoudstype. 

\textbf{Gepubliceerd:} Na het opslaan wordt de content gelijk zichtbaar gemaakt op de website.

\textbf{Revisie:} Elke keer nadat een content item is bijgewerkt, wordt er een nieuwe revisie aangemaakt.

\textbf{Moderatie revisie:} Indien deze optie is geactiveerd, is het mogelijk een revisie te wijzigen.

\begin{tabularx}{\textwidth}{ | p{5cm} |X|X|X|X| }
  \hline
  Inhoudstype & Gepubliceerd & Revisie & Moderatie revisie & Ge�mporteerd \\ \hline
  Agenda & N & J  & J & N  \\ \hline
  Bekendmaking & N & J  & J & J  \\ \hline
  Bericht  & J & N  & N & N  \\ \hline
  Bestand  & J & N  & N & N  \\ \hline
  Bestemmingsplan  & J & N  & N & J  \\ \hline
  Blog  & J & N  & N & N  \\ \hline
  Editorial  & J & J  & N & N  \\ \hline
  Eenvoudige pagina  & N & J  & J & N  \\ \hline
  FAQ  & N & J  & J & N  \\ \hline
  Forumonderwerp  & J & N  & N & N  \\ \hline
  Foto  & J & N  & N & N  \\ \hline
  Marktplaats  & J & N  & N & N  \\ \hline
  Nieuws  & N & J  & J & N  \\ \hline
  Onderwerp  & J & N  & N & N  \\ \hline
  Peiling  & N & J  & J & N  \\ \hline
  Persoon  & N & J  & J & N  \\ \hline
  Product  & J & N  & N & N  \\ \hline
  RSS  & J & N  & N & J  \\ \hline
  RSS Source  & N & N  & N & N  \\ \hline
  Regeling  & J & N  & N & J  \\ \hline
  Slide  & N & J  & J & N  \\ \hline
  VAC  & J & N  & N & J  \\ \hline
  Webformulier  & J & N  & N & N  \\ \hline
  Wiki  & J & N  & N & N  \\ \hline
\end{tabularx}