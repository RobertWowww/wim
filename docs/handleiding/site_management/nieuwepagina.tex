\subsection{Nieuwe (lege) pagina}\label{nieuwepagina}
Het aanmaken van een nieuwe pagina is nodig om een pagina op te bouwen uit Felixblokken. Denk heir bijvoorbeeld aan een nieuwsoverzicht- of smoelenboekpagina.

\textbf{Nieuwe pagina}

\begin{itemize}
\item Ga naar \drupalpath{admin/structure/empty-page/add} (rol: beheerder) en maakt een nieuw pad aan voor de nieuwe pagina. Titel mag leeg gelaten worden. Probeer bij het aanmaken van het pad zoveel mogelijk de menustructuur te volgen. Maak je een pagina onder wonen-en-leven aan, vul dan \emph{wonen-en-leven/nieuwepagina} in.
\item Na opslaan kom je op de overzichtslijst van alle lege pagina URL's. Ga nu naar de zojuist aangemaakt url toe. Dat kan via dit overzicht of door de URL in de browser in te typen.
\item Je komt nu op een lege pagina uit met alleen Felix regio's. Je kan nu elke regio vullen met een blok waar nodig. Zie \emph{Felix}\seeone{felix} voor een overzicht van Felixblokken.
\end{itemize}

\textbf{Koppelen aan een menu}
Om de pagina in een menu te zetten volg je de volgende stappen.

\begin{itemize}
\item Bewerk het hoofdmenu. Zie \emph{Menu}\seeone{menu} voor het bewerken en aanmaken van menuitems.
\item Voeg een nieuwe link toe.
\item Voeg een titel toe en vul de URL van de zojuist aangemaakte pagina op.
\item Kies bij Bovenliggend onderdeel waar deze pagina in de structuur terecht moet komen.
\item Sla op. Je komt nu terug op de lege pagina. Wanneer er menuitems zijn op hetzelfde niveau zal het submenu zichtbaar worden aan de linkerkant.
\end{itemize}