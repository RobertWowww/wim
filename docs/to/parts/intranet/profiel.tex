\subsection{Persoonlijk profiel}\label{profiel}

In het persoonlijk staan de volgende elementen:
\begin{itemize}
\item Foto
\item Algemene gegevens
\item Vrije tekstvelden
\begin{itemize}
\item Persoonlijke gegevens
\item Kennis/opleidingen/werk
\item Activiteiten/interesses
\item Overige informatie
\end{itemize}
\item Mijn projecten
\item Mijn producten
\item Mijn foto's
\item Links en downloads
\end{itemize}
De algemene gegevens zijn vooraf ingevuld en niet aanpasbaar.
Voor het aanpassen van de gegevens wordt gebruik gemaakt van de \usemodule{profile2} module i.c.m. de submodule \usemodule{profile2\_pages}. Hierin worden de volgende profielen aangemaakt:
\begin{itemize}
\item Profielfoto \\ Bevat een enkel filefield om een profielfoto te kunnen uploaden.
\\ Systeemnaam: \texttt{avatar}
\item Persoonlijk profiel \\ Bevat de vier vrije tekstvelden
\\ Systeemnaam: \texttt{personal}
\item Mijn projecten \\ Bevat een link veld (multiple) waarin gebruikers zelf links kunnen plaatsen.
\\ Systeemnaam: \texttt{projects}
\item Mijn producten \\ Bevat een link veld (multiple) waarin gebruikers zelf links kunnen plaatsen.
\\ Systeemnaam: \texttt{products}
\item Mijn foto's \\ Bevat een media veld (multiple)
\\ Systeemnaam: \texttt{photos}
\item Links en downloads \\ Bevat een link veld (multiple) waarin gebruikers zelf links kunnen plaatsen.
\\ Systeemnaam: \texttt{links}
\end{itemize}
Elk profiel type heeft een eigen pagina waarop de inhoud kan worden bewerkt. Deze pagina is beschikbaar op \texttt{profile-[name]/[uid]/edit}, waarbij \texttt{[name]} de systeemnaam is en \texttt{[uid]} de \emph{user id}. Wanneer men op het eigen profiel op \emph{Bewerken} klinkt dan komt men op \texttt{profile-personal/[uid]/edit}.

De \usemodule{profile2} module heeft een integratie met de \usemodule{views} module. Voor alle profiel types behalve persoonlijke informatie worden via de views module blokken gemaakt. Hierbij wordt de user id uit de URL gehaald via een contextual filter. In de footer van de view wordt via PHP code een link naar de bewerkpagina getoond indien het user id uit de url overeenkomt met de user id van de bezoeker.

De profielinformatie wat via de beschreven pagina's wordt aangepast blijft binnen Drupal en wordt niet verzonden naar andere systemen (zoals LDAP).

Als er extra velden toegevoegd moeten worden die vrij invulbaar zijn (voor bijv. de werktijden) dan kunnen deze toegevoegd worden aan het persoonlijk profiel.
