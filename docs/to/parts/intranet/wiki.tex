\subsection{Wiki}\label{wiki}

Voor de wiki functionaliteit wordt gebruik gemaakt van de \usemodule{wikitools} module. Tevens wordt een nieuw nodetype aangemaakt voor wikipagina's. Dit nodetype is gelijk aan een standaardpagina. Het pad (via \usemodule{pathauto}) is echter "wiki/[node:title]". Voor wikitools worden de volgende settings aangepast:
\begin{itemize}
\item Titel van voorpagina: Wiki
\item Node types: wiki
\item Wiki 404 type: Creation form
\end{itemize}
Voor alle overige settings worden de standaardwaarden gebruikt. De wiki is dan beschikbaar op \texttt{/wiki}. De voorpagina wordt tijdens de bouw aangemaakt.

In de bodytekst van wiki nodes kunnen links worden gebruikt naar \texttt{/wiki/Titel\_van\_pagina} (spaties dienen vervangen te worden door underscores). Bij het bezoeken van niet bestaande pagina's is een link beschikbaar om die pagina direct aan te maken.

Er wordt een custom module gemaakt die een input filter aanbiedt waarmee makkelijk links naar wiki pagina's kunnen worden gemaakt. Dit wordt uitgewerkt in een module met de naam \texttt{wiki\_links}.

Tekst voor filtering:
\begin{verbatim}
[[Titel van pagina]]
[[Titel van pagina|Titel van link]]
\end{verbatim}
Tekst na filtering:
\begin{verbatim}
<a href="/wiki/Titel_van_pagina">Titel van pagina</a>
<a href="/wiki/Titel_van_pagina">Titel van link</a>
\end{verbatim}
In het pad worden spaties vervangen door underscores.

De \texttt{freelinking} module biedt vrijwel identieke functionaliteit aan, maar werkt niet voor pagina's die niet bestaan en is daarom voor een wiki minder goed bruikbaar.

