\subsection{Teampagina}\label{teampagina}

De teampagina is een verzamelpunt van agenda items, berichten, veelgestelde vragen en documenten van een specifiek team. Hiervoor wordt gebruik gemaakt van subsites via \usemodule{dominion}. De subsites worden op een directory geplaatst, bijv. \texttt{/teams/communicatie}. De directory is tevens het pad van de teampagina, dat feitelijk de homepage is van de subsite. Toevoegen van inhoud moet ook vanaf de subsite, op bijv. \texttt{team/communicatie/node/add/event}.

\subsubsection{Blokken}

De landingspagina (voorpagina) van de team subsite wordt gemaakt met de \usemodule{emptypage} module. Deze is bij het aanmaken volledig blanco. Ook de rechterkolom is leeg. Via \usemodule{felix}\seeone{felix} kunnen de volgende blokken worden geplaatst:
\begin{itemize}
\item Laatste 5 agenda items
\item Laatste 5 berichten
\item Laatste 5 veelgestelde vragen
\item Laatste 5 documenten
\end{itemize}
De bedoeling is dat ieder team deze blokken naar eigen inzicht kan verdelen over de middenkolom en de rechterkolom.

De blokken worden gemaakt met \usemodule{views}. Om de theming hiervan onafhankelijk te maken van de blokken met laatste 5 nodes van het internet\seeone{felixcontenttypeblokken} wordt hiervoor een nieuwe view gemaakt. In de footer worden links naar de \texttt{node/add/\%} pagina's getoond indien de bezoeker daar toegang toe heeft. Dit moet gecontroleerd worden via de \texttt{dominion\_restrict} module:

\begin{verbatim}
global $user;
if (dominion_restrict_node_access('event', 'create', $user) {
  print l(t('Evenement toevoegen'), 'node/add/event');
}
\end{verbatim}

De lees meer link gaat naar een lijstweergave van alle items.

\subsubsection{Medewerkers lijst}

Onderaan de landingspagina wordt een lijst van medewerkers getoond. Deze lijst wordt gemaakt met de \usemodule{views} module. Dit is een lijst van profielen via \usemodule{profile2} module\seeone{profiel} van gebruikers die gekoppeld zijn aan het huidige domein. Aan deze view worden velden toegevoegd voor foto, naam, gebruikersrol en telefoonnummer.

\subsubsection{Beheer van team members}\label{teammemberbeheer}

Binnen de team subsites komt een aparte pagina voor het beheer van de leden in het team. Deze pagina is toegankelijk voor eindredacteuren, dus niet voor ieder teamlid. Hierin kunnen bestaande gebruikers toegevoegd worden aan het team, welke dan binnen de team subsite de rol \emph{team member} krijgen\seeone{rollen}. Met deze rol krijgt de gebruiker de mogelijkheid om inhoud direct te publiceren binnen de team subsite\seeone{workflow}. In dit onderdeel kunnen uiteraard ook personen uit het team worden gehaald. Er komt vooralsnog geen optie voor leden om zichzelf af te melden.

