\section{Modulereferentie}\label{tools}

\begin{multicols}{2}
%\printindex{modules} werkt niet.
\input{modules.ind}
\end{multicols}

\subsection{Externe libraries}

\begin{itemize}
 \item CKEditor
 \item Caroufredsel
\end{itemize}

\subsection{Frontend}

\subsubsection{Tools}
\begin{itemize}
 \item Ruby 1.8+
 \item Compass
 \item SASS
 \item Susy
\end{itemize}

\subsubsection{config.rb}
In de map themes/ccv/ komt dit config.rb bestand.
\begin{lstlisting}[language=C]
require "susy"

http_path = "../"
css_dir = "css"
sass_dir = "sass"
images_dir = "img"
javascripts_dir = "js"
fonts_dir = "font

project_type = :stand_alone
line_comments = false
preferred_syntax = :scss
 \end{lstlisting}
 
\subsubsection{Sprites}
Sprite map wordt geplaatst in themes/ccv/img/base/
\begin{lstlisting}[language=C]
$base-spacing: 100px;
@import "base/*.png";
@include all-base-sprites;
 \end{lstlisting}
 
\subsubsection{Grid}
Grids worden gerealiseerd met Susy.
\begin{lstlisting}[language=C] 
// variables
$total-columns: 12;
$column-width: 85px;
$gutter-width: 30px;
$grid-padding: $gutter-width;

$show-grid-backgrounds: false;

// grid definition
.container {
  @include container;
  @include susy-grid-background;
  padding: 0;
}

// per region grid bepalen
.ag1 { @include span-columns(4,12); } 
.ag2 { @include span-columns(8 omega,12); } 
// etc
 \end{lstlisting} 
 
 \subsubsection{Kleurgebruik}
De kleuren worden enkel in 100\% tint toegepast.

\begin{lstlisting}[language=C] 
$white: #FFFFFF;
$darkblue: #3C4A84;
$lightblue: #009DD8;
$green: #6BA634;
$black: #353535;
$darkgray: #6C6F70;
$lightgray: #C4C7C8;
$link_color: $lightblue;
$link_color_hover: $green;
\end{lstlisting} 

\subsubsection{Webfonts}
Webfonts worden gegenereerd door http://www.fontsquirrel.com/tools/webfont-generator.

Droid Sans is vrij verkrijgbaar via Google: http://www.google.com/fonts/specimen/Droid+Sans.
Museo Slab is een betaald font. Fonts.com heeft Museo Slab in zijn suite zitten.
\begin{lstlisting}[language=C] 
@font-face {
    font-family: 'droid_sansregular';
    src: url('../font/droidsans/droidsans-webfont.eot');
    src: url('../font/droidsans/droidsans-webfont.eot?#iefix') format('embedded-opentype'),
         url('../font/droidsans/droidsans-webfont.woff') format('woff'),
         url('../font/droidsans/droidsans-webfont.ttf') format('truetype'),
         url('../font/droidsans/droidsans-webfont.svg#droid_sansregular') format('svg');
    font-weight: normal;
    font-style: normal;
}
\end{lstlisting} 
 
\subsubsection{Responsive}
Via de at-breakpoint functie in Susy kunnen we bepalen wanneer de website een andere weergave heeft.
Voor het menu gebruiken we http://responsivemobilemenu.com/en/.

Het zoekformulier wordt door middel van jQuery toggle functie weergegeven.
Content in tabs blijven in tabs en wordt geen accordion zoals in het design staat.

\subsubsection{jQuery}
Accordion en tabs wordt gerealiseerd door middel van de volgende functie in preprocess\_html:
\begin{lstlisting}[language=C]
drupal_add_library('system', 'ui.accordion');
drupal_add_library('system', 'ui.tabs');
\end{lstlisting}

Content wordt weergegeven in verschillende tabs. De fields worden in de node template uitgelezen en wordt de juiste classes toegevoegd.

\subsubsection{Theme info}
\begin{lstlisting}[language=C] 
name = CCV
core = 7.x
description = "Theme voor CCV"

scripts[] = js/lib/jquery.carouFredSel-6.2.0-packed.js
scripts[] = js/script.js

stylesheets[all][] = css/all.css

regions[sidebar] = Sidebar
regions[content] = Content

regions[region-1-1] = Region 1-1
regions[region-1-2] = Region 1-2

regions[region-2-1] = Region 2-1
regions[region-2-2] = Region 2-2
regions[region-2-3] = Region 2-3

regions[region-3-1] = Region 3-1
regions[region-3-2] = Region 3-2
regions[region-3-3] = Region 3-3
regions[region-3-3] = Region 3-4
\end{lstlisting} 
 