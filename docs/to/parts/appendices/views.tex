\section{Views}\label{views}

\subsection{Agenda}
\subsubsection{Laatste Agenda items}\label{laatste-agenda-items}
View block gefilterd op Content Type Agenda en op gepubliceerd. Gesorteerd via Post Date. Weergave via View mode Teaser. Maximaal 3 items getoond. Inclusief 'Lees meer' link (naar node detailpagina) onder elk item.

\subsection{Nieuws}
\subsubsection{Nieuws Overzicht}\label{nieuws-overzicht}
View page gefilterd op Content Type Nieuws en op gepubliceerd. Gesorteerd via Post Date. Titel is Nieuwsarchief. 7 items getoond per pagina (met pager).

Getoonde velden:
\begin{itemize}
\item Afbeelding (indien aanwezig),
\item Titel,
\item Introductietekst.
\end{itemize}

\subsubsection{Laatste Nieuws items}\label{laatste-nieuws-items}
View block gefilterd op Content Type Nieuws en op gepubliceerd. Gesorteerd via Post Date. Weergave via View mode Teaser.
Maximaal 3 items getoond. Inclusief 'Lees meer' link (naar node detailpagina) onder elk item.

Getoonde velden:
\begin{itemize}
\item Titel,
\item Introductietekst (afgekort naar 80 karakters),
\item Lees meer link.
\end{itemize}

Attachment footer: link 'Meer nieuwsberichten' naar nieuws overzichtspagina.

\subsection{Bekendmakingen}
\subsubsection{Bekendmakingen op de kaart}\label{bekendmakingen-markers}

Middels \usemodule{location}, \usemodule{gmap} en \usemodule{gmap\_location} wordt een view met een page display opgezet die alle markers op de kaart laat zien.

Filters:
\begin{itemize}
\item Content Type: Bekendmaking
\item Published: Ja
\end{itemize}

Exposed filters:
\begin{itemize}
\item Titel (vrij in te vullen tekstveld),
\item Type bekendmaking,
\item Week,
\item Jaar,
\item Postcode (vrij in te vullen tekstveld)
\end{itemize}

Style options:
\begin{itemize}
\item Display tooltip: Content: Title
\item Data source: Location.module
\end{itemize}

\subsubsection{Bekendmakingen overzicht attachment}\label{bekendmakingen-overzicht}

View block gefilterd op Content Type Bekendmaking en op gepubliceerd. Weergave via View mode Teaser. Maximaal 7 items getoond met pager en lees meer link.

Deze view wordt gebruikt als attachment bij Bekendmakingen op de kaart \seeref{bekendmakingen-markers}.

\subsection{Blog}
View blog gefilterd op Content Type Blog en op gepubliceerd. Gesorteerd via Post Date. Titel is "Weblogs". 10 items getoond per pagina (met pager).

Getoonde velden: 
\begin{itemize}
\item Content type blog teaser
\end{itemize}

Filters
\begin{itemize}
\item Content Type: Blog
\item Published: Ja
\end{itemize}

Path
\begin{itemize}
\item /blog
\item /blog/\% (display voor blog van gebruiker)
\end{itemize}

Contextuele filter
\begin{itemize}
\item User uid
\end{itemize}

\subsection{Fotoalbum}
\subsubsection{Categorie\"{e}n}
View op tags uit de vocabulaire \emph{Tags images}. De view komt op \texttt{/fotoalbum}.
\subsubsection{Foto's}
View op images (file entity) met een specifieke tag. De view komt op \texttt{/fotoalbum/\%}, waarbij het tweede element de entity id is van de categorie.


\subsection{Overige onderwerpen carrousel}
View blok gefilterd op gepubliceerd en of node aanwezig is in de nodequeue. Weergave via View mode Full Node. Geen limiet op items geen pager. Een relatie wordt gelegd met de nodequeue.

Getoonde velden: 
\begin{itemize}
\item Content type Slide Full Node
\end{itemize}

Filters
\begin{itemize}
\item Published: Ja
\item (nodequeue) In queue: Ja
\end{itemize}

Blok
\begin{itemize}
\item Naam: Overige onderwerpen
\end{itemize}

Relatie
\begin{itemize}
\item Nodequeue
\item Limit to nodequeue: ondewerpen
\end{itemize}

\subsection{Smoelenboek}
View facebook op users (teasers). Gesorteerd op naam. Titel is "Smoelenboek". 10 items getoond per pagina (met pager).

Getoonde velden: 
\begin{itemize}
\item Pasfoto
\item Naam
\end{itemize}

Filters
\begin{itemize}
\item Rol: medewerker
\end{itemize}

Path
\begin{itemize}
\item /smoelenboek
\end{itemize}
