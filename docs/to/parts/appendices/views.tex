\section{Views}\label{views}

\subsection{Agenda}
\subsubsection{Laatste Agenda items}\label{laatste-agenda-items}
View block gefilterd op Content Type Agenda en op gepubliceerd. Gesorteerd via Post Date. Weergave via View mode Teaser. Maximaal 3 items getoond. Inclusief 'Lees meer' link (naar node detailpagina) onder elk item.

\subsection{Nieuws}
\subsubsection{Nieuws Overzicht}\label{nieuws-overzicht}
View page gefilterd op Content Type Nieuws en op gepubliceerd. Gesorteerd via Post Date. Titel is Nieuwsarchief. 7 items getoond per pagina (met pager).

Getoonde velden:
\begin{itemize}
\item Afbeelding (indien aanwezig),
\item Titel,
\item Introductietekst.
\end{itemize}

\subsubsection{Laatste Nieuws items}\label{laatste-nieuws-items}
View block gefilterd op Content Type Nieuws en op gepubliceerd. Gesorteerd via Post Date. Weergave via View mode Teaser.
Maximaal 3 items getoond. Inclusief 'Lees meer' link (naar node detailpagina) onder elk item.

Getoonde velden:
\begin{itemize}
\item Titel,
\item Introductietekst (afgekort naar 80 karakters),
\item Lees meer link.
\end{itemize}

Attachment footer: link 'Meer nieuwsberichten' naar nieuws overzichtspagina.

\subsection{Bekendmakingen}\label{bekendmakingenview}
\subsubsection{Bekendmakingen overzicht}\label{bekendmakingen-overzicht}

Er wordt een page display opgezet die de bekendmakingen in een lijst laat zien. Via een attached display wordt boven de lijst een kaart met  markers (van dezelfde bekendmakingen) geplaatst.

Fields:
\begin{itemize}
\item Content: Title
\item Location: Postal code
\item Location: Street
\item Content: Title \\
Het tweede titelveld wordt gebruikt als tooltip op de kaart en heeft daarom de volgende afwijkende settings:
\begin{itemize}
\item Create a label: uitgevinkt
\item Exclude from display: aangevinkt
\item Link this field to the original piece of content: uitgevinkt
\item Administrative title: Tooltip
\end{itemize}
\end{itemize}

Filters:
\begin{itemize}
\item Content Type: Bekendmaking
\item Published: Ja
\end{itemize}

Exposed filters:
\begin{itemize}
\item Titel (vrij in te vullen tekstveld)
\item Type bekendmaking (taxonomie, dropdown)
\item Status bekendmaking (taxonomie, checkboxes)
\item Jaar en maand (via \usemodule{date} module)
\item Postcode (vrij in te vullen tekstveld)
\end{itemize}

Sort criteria:
\begin{itemize}
\item Post date (exposed)
\item Title (exposed)
\end{itemize}

Style options:
\begin{itemize}
\item Display tooltip: Content: Title
\item Data source: Location.module
\end{itemize}

De exposed filters worden in een blok geplaatst. Bij de settings wordt \emph{expose sort order} uitgevinkt.

\subsubsection{Bekendmakingen kaart}\label{bekendmakingen-markers}

Middels \usemodule{location}, \usemodule{gmap} en \usemodule{gmap\_location} wordt een view met een page display opgezet. Deze display wordt gebruikt als attachment bij Bekendmakingen overzicht \seeref{bekendmakingen-overzicht}.

Format:
\begin{itemize}
\item Format: GMap
\begin{itemize}
\item Data source: location.module
\item Display a tooltip when hovering over markers: aangevinkt
\item Tooltip field: Tooltip
\end{itemize}
\end{itemize}

Attachment settings:
\begin{itemize}
\item Attach to: page
\item Attachment position: before
\item Inherit contextual filters: yes
\item Inherit exposed filters: yes
\end{itemize}

\subsection{Blog}
View blog gefilterd op Content Type Blog en op gepubliceerd. Gesorteerd via Post Date. Titel is "Weblogs". 10 items getoond per pagina (met pager).

Getoonde velden: 
\begin{itemize}
\item Content type blog teaser
\end{itemize}

Filters
\begin{itemize}
\item Content Type: Blog
\item Published: Ja
\end{itemize}

Path
\begin{itemize}
\item /blog
\item /blog/\% (display voor blog van gebruiker)
\end{itemize}

Contextuele filter
\begin{itemize}
\item User uid
\end{itemize}

\subsection{Fotoalbum}
\subsubsection{Categorie\"{e}n}
View op tags uit de vocabulaire \emph{Tags images}. De view komt op \texttt{/fotoalbum}.
\subsubsection{Foto's}
View op images (file entity) met een specifieke tag. De view komt op \texttt{/fotoalbum/\%}, waarbij het tweede element de entity id is van de categorie.


\subsection{Overige onderwerpen carrousel}
View blok gefilterd op gepubliceerd en of node aanwezig is in de nodequeue. Weergave via View mode Full Node. Geen limiet op items geen pager. Een relatie wordt gelegd met de nodequeue.

Getoonde velden: 
\begin{itemize}
\item Content type Slide Full Node
\end{itemize}

Filters
\begin{itemize}
\item Published: Ja
\item (nodequeue) In queue: Ja
\end{itemize}

Blok
\begin{itemize}
\item Naam: Overige onderwerpen
\end{itemize}

Relatie
\begin{itemize}
\item Nodequeue
\item Limit to nodequeue: ondewerpen
\end{itemize}

\subsection{Smoelenboek}
View facebook op users (teasers). Gesorteerd op naam. Titel is "Smoelenboek". 10 items getoond per pagina (met pager).

Getoonde velden: 
\begin{itemize}
\item Pasfoto
\item Naam
\end{itemize}

Filters
\begin{itemize}
\item Rol: medewerker
\end{itemize}

Path
\begin{itemize}
\item /smoelenboek
\end{itemize}
