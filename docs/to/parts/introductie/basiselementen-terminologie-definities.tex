\section{Basiselementen, terminologie en definities}

Elk softwaresysteem hanteert een bepaald taalgebruik en termen. Dit om een eenduidige betekenis van een term te defini\"eren. Voor het schrijf- en leesgemak voor de ontwikkelaars worden dan ook de termen gebruikt die bij Drupal gangbaar zijn. Echter voor een niet-ontwikkelaar of een niet-Drupal gebruiker kan dit verwarring opleveren. 

\begin{description}

\item[Drupal] Drupal is het gekozen Content Management Systeem dat als framework dient voor de bouw van de nieuwe website. Drupal is een Open Source CMS dat geschreven is in PHP. 

\item[Modules] Drupal is een modulair gebaseerd systeem. Een module is een discrete verzameling code die als uitbreiding dient. In andere systemen wordt dit ook wel een plug-in of extensie genoemd. 

\item[Contrib (contributed) modules] Wanneer gesproken wordt over contrib of gecontribueerd modules, wordt hiermee bedoeld de beschikbare modules op drupal.org. Deze modules zijn beschikbaar gesteld door communityleden, en zijn vrij te verkrijgen. 

\item[Custom modules] Hiermee wordt bedoeld modules die op maat zijn gemaakt omdat de functionaliteit niet beschikbaar is binnen contrib modules. Een custom module kan ook een uitbreiding zijn op een bestaande contrib module. 

\item[Contenttype] Een contenttype is een term die veelgebruikt wordt in Drupal. Het is een type pagina met eenzelfde structuur van velden. Een contenttype is bedoeld om onderscheid te kunnen maken tussen de inhoud. 

\item[Taxonomy] Voor de categorisering van inhoud wordt gebruik gemaakt van taxonomie. Taxonomie is de term die gebruikt wordt om labels (tags) aan inhoud te koppelen. Dit kunnen vooraf ingestelde lijsten zijn of vrij in te vullen door de gebruiker. 

\item[User] Een gebruiker binnen Drupal.

\item[Roles] Drupal heeft een op rollen gebaseerd permissiesysteem. De rollen zijn configureerbaar en uit te breiden naar wens. 
De standaard rollen die binnen Drupal ingesteld zijn:
\begin{itemize}
\item[Anonymous users] Niet-ingelogde gebruikers, bezoekers.
\item[Autenticated users] Ingelogde gebruikers.
\item[Administrator] De rol voor de beheerder van de site. Deze rol mag feitelijk alle pagina's en inhoud bezoeken, alsmede alle instellingen van een website aanpassen. 
\end{itemize}
\item[Blok] Drupal maakt veel gebruik van blokken (blocks). Een blok is een los element, waarbij de inhoud van een blok verschillend kan zijn. Een blok kan geplaatst worden op een pagina. Afhankelijk van de implementatie kunnen blokken ook dynamisch getoond worden op basis van de context van de pagina. 

\item[Views] Een veelgebruikte contrib module, die wordt gebruikt om overzichten en lijsten te maken (views). Als er gesproken wordt over "een view" dan wordt er verwezen naar een afzonderlijke implementatie van de views module. 

\item[Theme] Een verzameling van bestanden (.PHP, .INFO, .CSS, .JPG, .GIF, .PNG), die samen het uiterlijk van de website bepalen. Een theme bevat elementen zoals een header, iconen, de indeling van het blokken systeem. 

\item[Template] \'{E}\'{e}n bestand dat bestaat uit HTML in combinatie met PHP code, bedoelt om een specifieke structuur te leveren. 

\item[Cache] Drupal genereert (onderdelen van) pagina's en slaat deze op in de cache. Door dit systeem hoeft een pagina niet telkens opnieuw opgebouwd te worden, wat de snelheid ten goede komt. 

\item[CMS] Afkorting: Content Management Systeem, hiermee wordt bedoeld het beheergedeelte van Drupal. 

\item[Core] Het basissysteem inclusief de basismodules van Drupal. 

\item[Cron] Een script dat gebruikt wordt om automatisch bepaalde routines uit te voeren. Drupal gebruikt een cronjob om regelmatig terugkerende routines uit te voeren. 

\item[Entity] Een entity is een verzameling van informatie die niet specifiek inhouds gebonden is. Een user kan bijvoorbeeld een entity zijn. 

\item[Field] Een field kan een onderdeel zijn van een contenttype of entity. Hiermee kan informatie worden toegevoegd aan een entiteit. 

\item[Menu] Een functionaliteit die vaak gebruikt wordt om de navigatie mee in te richten. Een menu item is een link in de navigatie. 

\item[Node] Met Node wordt bedoeld een enkel stuk content van een contenttype dat door middel van een node-id enkelvoudig wordt beschreven. Een node kan een pagina hebben met een eigen url. 

\item[Regio] Een regio is een bepaald gedefin\"eerd gedeelte van de website. In een regio kan inhoud worden geplaatst, alsmede blokken.

\end{description}