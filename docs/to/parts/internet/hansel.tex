\subsection{Hansel}

De module \usemodule{hansel} is oorspronkelijk gemaakt om flexibel de breadcrumbs aan te kunnen passen. Hansel kan (en zal) ingezet worden voor de volgende functionaliteiten:
\begin{itemize}
\item Breadcrumbs
\item Vriendelijke URL-paden
\item Activeren van menu-items
\end{itemize}
Het eerste item is "Home" (wordt niet meegenomen in de URL-paden). Het laatste item is de huidige pagina en wordt getoond als platte tekst (dus geen link). De volgende settings worden gebruikt (onder \emph{Settings}-tab):
\begin{itemize}
\item Render last item as a link: uitgeschakeld
\item Hide last item: uitgeschakeld
\item Maximum length for individual items: 100
\item Trim on word boundary: ingeschakeld
\item Maximum number of items: 0 (niet gelimiteerd)
\item Remove first item from hansel path token: ingeschakeld
\item Cache: 10 minuten
\item Cache whole breadcrumbs: uitgeschakeld
\item Set active menu item: ingeschakeld
\item Skip first crumb: ingeschakeld
\item Set active menu name for current path: uitgeschakeld
\item Lookup url aliases when looking up parents: uitgeschakeld
\end{itemize}
De rules worden als volgt ingesteld:
\begin{itemize}
\item start \\ add link \emph{Home}, switch on \emph{domain id}
\begin{itemize}
\item \textless default\textgreater \\  add parents, switch on url argument 1
\begin{itemize}
\item \textless default\textgreater \\ add link to current page, leave
\item node \\ add link to node, leave
\end{itemize}
\end{itemize}
\end{itemize}
Voor de switch op \emph{domain id} wordt de submodule \usemodule{hansel\_domain} aangezet.
Later in het project zullen onder het tabblad \emph{Nodetypes} de bijbehorende pagina's ingesteld worden voor de nodetypes die doorgaans niet in het menu worden geplaatst, zoals nieuwsberichten.

