\subsection{Zoeken}\label{zoeken}
Voor de zoekfunctionaliteit zullen we gebruik maken van de \usemodule{apachesolr} module. We gaan uit van default settings (dat wil zeggen: configuratiebestanden die meegeleverd worden bij de \texttt{apachesolr} module). Berekeningen en settings van relevantie, synoniemenlijsten etc. zullen we ongemoeid laten. Het is vaak ook niet noodzakelijk om dit aan te passen aangezien de default settings doorgaans goede resultaten leveren.

\subsubsection{Zoekbox in header}
Een zoekopdracht invoeren gaat altijd via de zoekbox in de header. Voor dit blok wordt de standaard \usemodule{search} module uit Drupal core gebruikt.

\subsubsection{Zoekresultaten}
De zoekresultaten worden getoond titel en snippet. De zoekwoorden zullen we cursief maken.

\subsubsection{Zoeken in bijlagen}
Bijlagen die zijn toegevoegd aan nodes worden meegenomen bij het indexeren. Hiervoor wordt \emph{Apache Tika} i.c.m. de \usemodule{apachesolr\_attachments} module gebruikt. Tika kan diverse formaten omzetten naar platte tekst die geschikt is voor indexatie. Hieronder vallen ook PDF en Microsoft Office formaten\footnote{http://tika.apache.org/1.4/formats.html}.

\subsubsection{Bias settings}
De \usemodule{apachesolr} module biedt standaard Bias settings aan voor velden en types. Deze settings zijn beschikbaar voor de admin user. De mogelijkheden zullen worden beschreven in de handleiding.

\subsubsection{Spellingscontrole}
De optie \emph{spellingscontrole} wordt ingeschakeld (onder de settings van de zoekpagina). De woordenlijst komt standaard uit de ge\"{i}ndexeerde tekst. Later kan - indien wenselijk - een eigen woordenlijst worden toegevoegd. Die zal echter generiek zijn voor alle Dimpact gemeenten en kan niet via de admin worden toegevoegd.

\subsubsection{Zoeken binnen landelijke voorzieningen}


