\subsection{Agenda \& Evenementen}\label{agenda-en-evenementen}

\subsubsection{Overzicht}\label{agendaoverzicht}

De overzichtspagina van de agenda is een lijstpagina met alle projecten die beginnen vanaf de actuele dag, gesorteerd op begindatum. In de rechter zijbalk wordt een blok toegevoegd met langlopende evenementen. Dit zijn evenementen die eerder dan de actuele dag zijn begonnen, maar nog steeds bezig zijn.

\subsubsection{Agenda detailpagina}\label{agenda-detail}

Een agenda item bevat de volgende velden die op de node detailpagina worden getoond:
\begin{itemize}
\item Titel
\item Datum (inclusief tot-datum indien beschikbaar)
\item Tijd
\item Toegang
\item Locatie
\item Afbeelding
\item Body tekst
\end{itemize}
Alleen de tekstuele locatie wordt getoond (indien beschikbaar). De co\"{o}rdinaten zijn wel in te voeren. De hier beschreven velden zijn de velden die op de detailpagina worden getoond, dus niet de referentie van beschikbare velden\seeone{sec:content-event}.

In de sidebar aan de rechterkant worden eveneens enkele velden ontsloten m.b.v. de \usemodule{cck\_blocks} module:
\begin{itemize}
\item Formulieren (nodereference),
\item Externe links (link field).
\end{itemize}

\subsubsection{Agenda teaser blok}

Het blok is een \emph{view} die een \emph{teaser view} laat zien van de laatste drie items van het type Agenda, gesorteerd op aanmaakdatum. Zie \seeref{laatste-agenda-items}

De teaser view bevat:
\begin{itemize}
\item Titel,
\item Datum (optioneel met tot-datum)
\item en "Lees meer" link naar node detailpagina.
\end{itemize}