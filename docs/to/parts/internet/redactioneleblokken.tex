\subsection{Redactionele blokken}\label{felix}

De module \usemodule{felix} wordt ingezet om het voor redacteuren mogelijk te maken om redactionele blokken te plaatsen binnen vooraf gedefinieerde regio's. Welke regio's dat zijn wordt in deze sectie verder uitgewerkt. Alle blokken die worden geplaatst zijn specifiek voor \'{e}\'{e}n pagina en worden dus niet automatisch op meerdere (gerelateerde) pagina's geplaatst. De blokken zijn wel generiek over alle subsites. Wanneer een blok op \texttt{/nieuws} wordt geplaatst dan zal deze op een andere subsite ook zichtbaar zijn op dat pad, mits de geplaatste node ook op dat domein is gepubliceerd.

Onderstaand tabel geeft een overzicht van de Felix regio's.

\begin{tabularx}{\linewidth}{| p{3cm} | p{3cm} | X | p{3cm} | }
\hline
\rowcolor{tableheader}
\textbf{Naam} & \textbf{Systeemnaam} & \textbf{Verschillend per} & \textbf{Positie} \\ \hline
Paginaspecifiek & sidebar\_path & path & rechter zijbalk \\ \hline
Paginatype & sidebar\_node & nodetype & rechter zijbalk \\ \hline
Inhoud & content & path & onder de content \\ \hline
Footer kolom 1 & footer1 & domain & footer \\ \hline
Footer kolom 2 & footer2 & domain & footer \\ \hline
Footer kolom 3 & footer3 & domain & footer \\ \hline
\end{tabularx}

Om de blokken per domein te kunnen scheiden wordt gebruik gemaakt van de \usemodule{felix\_domain} submodule. In de rechterkolom worden twee regio's gebruikt. Blokken die geplaatst worden in de \emph{Paginaspecifiek} regio zijn specifiek voor die pagina. De \emph{Paginatype} regio is verschillend per nodetype. Als daarin een blok wordt geplaatst op een bekendmaking detailpagina dan zal deze op alle detailpagina's van bekendmakingen te zien zijn. De volgorde van de blokken kan alleen worden aangepast binnen de regio. Blokken die specifiek zijn voor de pagina staan altijd onder de blokken die generiek zijn voor het nodetype. Hierop wordt het gewicht van de regio's aangepast.

Er wordt \'{e}\'{e}n \emph{blockset} aangemaakt. De blokken die vrij te plaatsen zijn kunnen dus in elke regio worden gezet. De theming is wel afhankelijk van de regio (wordt volledig in CSS geregeld).

\subsubsection{Blokken per content type}\label{felixcontenttypeblokken}

Voor alle nodetypes die een detailpagina hebben wordt een blok aangemaakt met links naar de laatste 5 gepubliceerde items. Dit blok kan binnen elke felix regio redactioneel worden ingesteld.

