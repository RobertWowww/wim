\subsection{Unit Tests}

Bij elke test wordt zorggedragen voor een juiste installatie van configuratie, modules en permissies. Iedere Unit Test bestaat uit besloten functionaliteit. Meerdere tests worden gebruikt wanneer meerdere componenten als \'{e}\'{e}n geheel getest dienen te worden. In dat geval worden \emph{dependencies} gebruikt.

Een test bevat altijd de volgende functies:
\begin{itemize}
\item \texttt{getInfo}. Deze methode geeft de naam, beschrijving en de groep waar deze unit test deel van uitmaakt.
\item \texttt{setUp}. Deze methode definieert dependencies die nodig zijn om de tests succesvol te kunnen draaien. Denk aan het aanzetten van modules die noodzakelijk zijn, het aanmaken en inloggen van de testgebruiker, permissies etc.
\item \texttt{testSimpleTextX}. Deze methode bevat de test zelf. Hierin wordt duidelijk genoteerd wat de test behoort te doen. Bij oplevering zal aan alle vermelde condities worden voldaan.
\end{itemize}

Om de noodzakelijke configuratie voor de Unit Tests te beheren, wordt een feature gemaakt waarin alle dependencies worden opgenomen. Het aanzetten van deze feature heeft een complete installatie van het \projectname  project als resultaat.


