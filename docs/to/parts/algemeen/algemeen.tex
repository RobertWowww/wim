\section{Algemeen}\label{algemeen}

\subsection{Webrichtlijnen}

Opgeleverd werk voldoet volledig aan de HTML5 en CSS3 standaard zoals opgesteld door het \emph{World Wide Web consortium}\footnote{http://www.w3.org/standards/webdesign/htmlcss}.

\subsection{Browsers}
Deze paragraaf beschrijft op welke browsers wordt getest bij de bouw. Bij de bouw worden de \emph{best practices} aangehouden waardoor de functionaliteit op meer browsers zal werken dan in dit onderdeel aangegeven.

\subsubsection{Legenda voor desktop browsers en mobile browsers}
In de volgende 2 tabellen worden in de verschillende vakjes, symbolen gebruikt. Deze symbolen hebben een betekenis die in de onderstaande tabel wordt toegelicht.

\begin{tabularx}{\linewidth}{| p{5cm} | X |}
\hline
\rowcolor{tableheader}
\textbf{Symbool} & \textbf{Uitleg} \\ \hline
++	& 100\% stijlen en functionele implementatie \\ \hline
+	& Visueel goed werkend, minimale afwijking   \\ \hline
0	& Functioneel goed werkend, acceptabele      \\ \hline
-	& Strevend functioneel werkend, stijl niet optimaal  \\ \hline
N/A	& Niet van toepassing  \\ \hline
\end{tabularx}

\subsubsection{Desktop browsers}
\begin{tabularx}{\linewidth}{| X | l | l | l | l | l | l l |}
\hline
\rowcolor{tableheader}
��������\textbf{OS}���      & \textbf{IE10} & \textbf{IE9} & \textbf{IE8} & \textbf{Firefox} & \textbf{Chrome} & \textbf{Safari} & \textbf{Opera} \\ \hline
��������Windows 8���������� & ++� & N/A� & N/A & ++����� & ++���� & 0����� & 0���� \\ \hline
��������Windows 7���������� & ++� & ++� & N/A� & ++����� & ++���� & 0����� & 0���� \\ \hline
��������Windows Vista������ & N/A� & ++� & +� & ++����� & ++���� & 0����� & 0���� \\ \hline
��������Windows XP��������� & N/A & N/A & +� & ++����� & 0���� & 0����� & 0���� \\ \hline
��������Windows 2003 Server & N/A & N/A & N/A & ++����� & 0����� & 0����� & 0���� \\ \hline
��������Linux�������������� & N/A� & N/A & N/A & -������ & -��� & -����� & -���� \\ \hline
��������Mac OS������������� & N/A� & N/A & N/A & ++����� & ++���� & ++���� & 0���� \\ \hline
\end{tabularx}

\subsubsection{Mobiele browsers}

\begin{tabularx}{\linewidth}{| X | l | l | l | l | l | l |}
\rowcolor{tableheader}
\hline
��������\textbf{OS}��  & \textbf{Safari} & \textbf{Android} & \textbf{Blackberry} & \textbf{Chrome} & \textbf{Firefox} & \textbf{IEMobile} \\ \hline
��������iOS����������� & ++���� & N/A������������ & N/A������� & +��� & N/A���������� & N/A����� \\ \hline
��������Android������� & N/A��� & ++������������� & N/A������� & +��� & +���������� & N/A����� \\ \hline
��������Blackberry���� & N/A��� & N/A������������ & +�������� & N/A��� & N/A���������� & N/A����� \\ \hline
��������Windows Mobile & N/A��� & N/A������������ & N/A������� & N/A��� & N/A���������� & ++������ \\ \hline
��������Overig�������� & -����� & -�������������� & -��������� & -����� & -������������ & -������� \\ \hline
\end{tabularx}



\subsection{Taal}\label{taal}
De standaardtaal voor Drupal is Engels. De \usemodule{locale} module (uit Drupal core) zal worden ingezet om andere vertalingen te kunnen laten zien. De vertalingen van Drupal zal ge\"{i}mporteerd worden voor de modules waarvoor deze beschikbaar is. De vertalingen kunnen in het beheergedeelte aangevuld c.q. aangepast worden.

Bij oplevering zullen we de volgende talen aanmaken en importeren:
\begin{itemize}
\item Nederlands
\item Engels
\item Duits
\item Frans
\item Italiaans
\end{itemize}

\subsubsection{Datumformaat}
Het standaard datumformaat zullen we instellen op "dd maand yyyy" (bijv. 4 mei 2011).

\subsection{Rechten en rollen}\label{rollen}

zie \emph{Design briefing} vanaf pagina 7.

De volgende rollen worden aangemaakt:
\begin{itemize}
\item eindredacteur
\item redacteur
\end{itemize}

\subsection{Anti-spam}\label{antispam}

Er zullen geen specifieke maatregelen worden getroffen om spam tegen te gaan. 

\subsection{Cookies}\label{cookies}

Om gebruik te mogen maken van Google Analytics en Social Media zal een cookiebalk ingezet worden\seeone{cookiebalk}.
