\section{Algemeen}\label{algemeen}

\subsection{Webrichtlijnen}

Opgeleverd werk voldoet volledig aan de HTML5 en CSS3 standaard zoals opgesteld door het \emph{World Wide Web consortium}\footnote{http://www.w3.org/standards/webdesign/htmlcss}.

\subsection{Browsers}
Deze paragraaf beschrijft op welke browsers wordt getest bij de bouw. Bij de bouw worden de \emph{best practices} aangehouden waardoor de functionaliteit op meer browsers zal werken dan in dit onderdeel aangegeven.

\subsubsection{Legenda voor desktop browsers en mobile browsers}
In de volgende 2 tabellen worden in de verschillende vakjes, symbolen gebruikt. Deze symbolen hebben een betekenis die in de onderstaande tabel wordt toegelicht.

\begin{tabularx}{\linewidth}{| p{5cm} | X |}
\hline
\rowcolor{tableheader}
\textbf{Symbool} & \textbf{Uitleg} \\ \hline
++	& 100\% stijlen en functionele implementatie \\ \hline
+	& Visueel goed werkend, minimale afwijking   \\ \hline
0	& Functioneel goed werkend, acceptabele      \\ \hline
-	& Strevend functioneel werkend, stijl niet optimaal  \\ \hline
N/A	& Niet van toepassing  \\ \hline
\end{tabularx}

\subsubsection{Desktop browsers}
\begin{tabularx}{\linewidth}{| X | l | l | l | l | l | l l |}
\hline
\rowcolor{tableheader}
��������\textbf{OS}���      & \textbf{IE10} & \textbf{IE9} & \textbf{IE8} & \textbf{Firefox} & \textbf{Chrome} & \textbf{Safari} & \textbf{Opera} \\ \hline
��������Windows 8���������� & ++� & N/A� & N/A & ++����� & ++���� & 0����� & 0���� \\ \hline
��������Windows 7���������� & ++� & ++� & N/A� & ++����� & ++���� & 0����� & 0���� \\ \hline
��������Windows Vista������ & N/A� & ++� & +� & ++����� & ++���� & 0����� & 0���� \\ \hline
��������Windows XP��������� & N/A & N/A & +� & ++����� & 0���� & 0����� & 0���� \\ \hline
��������Windows 2003 Server & N/A & N/A & N/A & ++����� & 0����� & 0����� & 0���� \\ \hline
��������Linux�������������� & N/A� & N/A & N/A & -������ & -��� & -����� & -���� \\ \hline
��������Mac OS������������� & N/A� & N/A & N/A & ++����� & ++���� & ++���� & 0���� \\ \hline
\end{tabularx}

\subsubsection{Mobiele browsers}

\begin{tabularx}{\linewidth}{| X | l | l | l | l | l | l |}
\rowcolor{tableheader}
\hline
��������\textbf{OS}��  & \textbf{Safari} & \textbf{Android} & \textbf{Blackberry} & \textbf{Chrome} & \textbf{Firefox} & \textbf{IEMobile} \\ \hline
��������iOS����������� & ++���� & N/A������������ & N/A������� & +��� & N/A���������� & N/A����� \\ \hline
��������Android������� & N/A��� & ++������������� & N/A������� & +��� & +���������� & N/A����� \\ \hline
��������Blackberry���� & N/A��� & N/A������������ & +�������� & N/A��� & N/A���������� & N/A����� \\ \hline
��������Windows Mobile & N/A��� & N/A������������ & N/A������� & N/A��� & N/A���������� & ++������ \\ \hline
��������Overig�������� & -����� & -�������������� & -��������� & -����� & -������������ & -������� \\ \hline
\end{tabularx}



\subsection{Taal}\label{taal}
De standaardtaal voor Drupal is Engels. De \usemodule{locale} module (uit Drupal core) zal worden ingezet om andere vertalingen te kunnen laten zien. De Nederlandse vertalingen van Drupal worden ge\"{i}mporteerd voor de modules waarvoor deze beschikbaar is. De vertalingen kunnen in het beheergedeelte aangevuld c.q. aangepast worden.

\subsubsection{Datumformaat}
Het standaard datumformaat stellen we in op "dd maand yyyy" (bijv. 4 mei 2011).

\subsection{Rechten en rollen}\label{rollen}

De volgende rollen worden aangemaakt:
\begin{itemize}
\item eindredacteur
\item redacteur
\item medewerker
\item team member
\end{itemize}
Medewerkers krijgen rechten om inhoud aan te maken. Dit is relevant voor het intranet, waar iedereen op moet kunnen publiceren.

\subsection{Workflow}\label{workflow}
Voor deze workflow wordt de \usemodule{workbench} module i.c.m. \usemodule{workbench\_moderation} ingezet. Content kan de volgende statussen hebben:
\begin{itemize}
\item klad (\texttt{draft})
\item aangeboden ter goedkeuring (\texttt{needs review})
\item gepubliceerd (\texttt{published})
\end{itemize}
Inhoud met de status \texttt{draft} of \texttt{needs review} is alleen te zien voor redacteuren. De transitie van \texttt{draft} naar \texttt{needs review} kan gemaakt worden door redacteuren op eigen content, of door eindredacteuren op alle content. De transitie naar \texttt{published} kan alleen gemaakt worden door eindredacteuren. De workflow geldt voor zowel nieuwe inhoud als wijzigingen op reeds gepubliceerde inhoud.

Op het intranet wordt content vooral ingedeeld in teams. De teampagina's zijn losse subsites\seeone{teampagina}. Voor de team members wordt een nieuwe rol aangemaakt welke - qua workflow - dezelfde rechten heeft als een eindredacteur. Gebruikers met deze rol kunnen dus onbeperkt publiceren binnen de team subsite. De rechten voor overige functionaliteiten kunnen afwijken. Zo kunnen eindredacteuren gebruikers toevoegen aan een team, terwijl team members dit zelf niet kunnen.

Bij het aanmaken van inhoud wordt de publicatieoptie standaard op \texttt{published} gezet wanneer de gebruiker het recht heeft om dat item direct te publiceren. Men hoeft dus niet daarna de eigen content goed te keuren.

De workflow wordt op alle contenttypes aangezet. Tevens wordt een dashboard ingericht waar onder andere een lijst terug te vinden is met inhoud die goedgekeurd dient te worden\seeone{workbenchdashboard}.

Bij de gemeente specifieke implementatie kunnen rollen worden toegevoegd. Dit kan de gemeente niet zelf doen aangezien dit een hoge mate van kennis over Drupal vereist.

\subsection{Anti-spam}\label{antispam}

Er zullen geen specifieke maatregelen worden getroffen om spam tegen te gaan. 

\subsection{Cookies}\label{cookies}

Om gebruik te mogen maken van Google Analytics en Social Media zal een cookiebalk ingezet worden\seeone{cookiebalk}.
