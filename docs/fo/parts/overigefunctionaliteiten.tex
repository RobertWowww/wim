
\section{Overige functionaliteiten}
\label{sec:overigefunctionaliteiten}

\subsection{WYSIWYG}
We gebruiken een \emph{What You See Is What You Get} editor. DIt is een gebruiksvriendelijke manier van content invoeren in het CMS. De volgende opties zijn aanwezig:

\begin{itemize}
  \item Vet gedrukt
  \item Schuin
  \item Opsommings lijst
  \item Genummerde lijst 
  \item Definitie lijst
  \item Ongedaan maken
   \item Citaat
  \item Citaatblok
  \item Speciale tekens
  \item Kiezen uit opmaak
  \item Tabel
  \item Zoek en vervang
  \item Afkorting
  \item Acroniem
  \item Taal veranderen per woord
  \item Link invoegen
  \item Anker invoegen
  \item Media Invoegen
  \item Plakken speciaal
%  \item Invoegen van woord   % <ins>
%  \item Verwijderen van woord   % <del>
\end{itemize}

\subsection{Fotoalbum}
Een verzameling van gegroepeerde afbeeldingen. Gebruikers kunnen afbeeldingen uploaden en voorzien van een tag. Aan het uploaden van bestanden zitten restricties van zowel grootte als omvang. De naam van het bestand zal doormiddel van \emph{transliteration} aangepast worden naar een nette naam. Hierbij worden leesteken en spaties vervangen door koppeltekens. Van de geuploade bestanden worden kleinere versies en/of uitsnedes gemaakt.

\subsection{AddThis}
Om content te kunnen delen maken we gebruik van de functionaliteit die AddThis levert. De configuratie van deze functionaliteit is eenvoudig aan te passen.

\subsection{Readspeaker}
De website zal gebruikmaken van de voorleesfunctie van Readspeaker. De functie zal aanwezig zijn op elke pagina.

\subsection{RSS}
De content die in de feeds wordt opgenomen wordt bepaald aan de hand van het veld \emph{Tonen op website} dat in een aantal content types aanwezig is. De volgorde van de content is steeds op chronologische volgorde.

\subsection{Sitemap}
Voor eindgebruikers zal er een sitemap aanwezig zijn. Deze pagina toont een overzicht van de beschikbare pagina's op de site zoals deze ingericht zijn in het menu-systeem van Drupal.

\subsection{Prikbord}
Via het content type \emph{Marketplace} kan een ingelogde gebruiker van het intranet een advertentie plaatsen. De velden titel, body en categorie kunnen daar worden ingevuld. 

\subsection{Blog}
Een ingelogde gebruiker van het intranet kan een blog maken door middel van nodes aan te maken van het content type \emph{blog}. Er komt een overzichtspagina per gebruiker waar de blog te lezen is. Er is tevens een overzichtspagina van alle blogs van alle gebruikers. Het is mogelijk om een titel, body en tags toe te voegen aan het blog.

\subsection{Responsive layout}
De website zal responsive opgezet worden. Waarbij middels verschillende breekpunten de layout van de website aangepast wordt aan de breedte van het medium. In de smalste versie van de site zullen alle kolommen (en de blokken daarbinnen) onder elkaar getoond worden. Waarbij de volgorde; linker-, midden- en rechterkolom aangehouden wordt van boven naar beneden. Op artikelpagina's zal de belangrijkste content (het artikel zelf) bovenaan getoond worden. Hier zal de volgorde van boven naar beneden dus midden-, linker- en rechterkolom.

Bij bredere versies van de site zal de middenkolom schalen en zullen de rechter- en linkerkolom dezelfde breedte houden.