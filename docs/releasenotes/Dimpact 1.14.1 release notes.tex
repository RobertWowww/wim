% latex packages to load.
\documentclass[12pt]{article}
\usepackage{geometry}
\geometry{a4paper}
%\geometry{landscape}
\usepackage[parfill]{parskip}    % Activate to begin paragraphs with an empty line rather than an indent
\usepackage{graphicx}
\usepackage{amssymb}
\usepackage{epstopdf}
\usepackage{fancyhdr}
\usepackage{fullpage}
\usepackage{appendix}
\usepackage{newclude}
\usepackage{datetime}
\usepackage{hyperref}
\usepackage{color}
\usepackage{multicol}
\usepackage{tabularx}
\usepackage{enumerate}
\usepackage{enumitem}
\usepackage{listings}
\usepackage{varwidth}
\usepackage{wallpaper}
\usepackage{lastpage}
\usepackage{titling}
%\usepackage{multind}
\usepackage[dutch]{babel}
\usepackage[table]{xcolor}
\usepackage{mathtools}
\usepackage{amsmath}
\usepackage{MnSymbol}
\usepackage{wasysym}

\usepackage[default,osfigures,scale=0.90]{opensans} %% Alternatively
%% use the option 'defaultsans' instead of 'default' to replace the
%% sans serif font only.
\usepackage[T1]{fontenc}

%tikz images, unquote to use block, color block, autograph block, cloud, line and line node
\usepackage[framemethod=tikz]{mdframed}
\usepackage{tikz}
\usetikzlibrary{shapes, arrows}

\tikzstyle{block} = [rectangle, draw, text width = 6em, text centered, rounded corners, minimum height = 4em]
\tikzstyle{colorblock} = [rectangle, draw, fill = blue!20, text width = 6em, text centered, rounded corners, minimum height = 4em]
%\tikzstyle{inputblock} = [rectangle, draw, text width = 24em, minimum height = 2.5em]
%\tikzstyle{autographblock} = [rectangle, draw, fill = black!1, text height = 0, text depth = 2cm, text width = 24em, minimum height = 8em]
\tikzstyle{cloud} = [ellipse, draw, minimum height = 4em]
\tikzstyle{line} = [draw, -latex']
\tikzstyle{line node} = [draw, fill = white]
\tikzstyle{decision} = [diamond, aspect=2, draw, text badly centered]

\usepackage[official]{eurosym}

\newcommand{\usemodule}[1]{\index{modules}{#1}\texttt{#1}}

\definecolor{gray}{rgb}{0.5,0.5,0.5}
\definecolor{tableheader}{rgb}{0.7,0.7,0.7}

\setlist[description]{style=nextline}
\renewcommand{\familydefault}{\sfdefault}

% link setup
\hypersetup{
    colorlinks,
    citecolor=black,
    filecolor=black,
    linkcolor=black,
    urlcolor=black,
}

\DeclareGraphicsRule{.tif}{png}{.png}{`convert #1 `dirname #1`/`basename #1 .tif`.png}

% usefull commands:
\newcommand{\seeref}[1]{\ref{#1} p.\pageref{#1}}
%\newcommand{\see}[1]{ (zie \ref{#1} p.\pageref{#1})}
\newcommand{\seesee}[2]{ (zie \ref{#1} p.\pageref{#1},  \ref{#2} p.\pageref{#2})}

% style for code blocks
\lstset{
    linewidth=1\textwidth,
    breaklines=true,
    numbers=left,                   % where to put the line-numbers
    numberstyle=\tiny\color{gray},  % the style that is used for the line-numbers
    stepnumber=1,                   % the step between two line-numbers. If it's 1, each line 
    numbersep=5pt, 
    basicstyle=\footnotesize,
}

% Header and Footer settings
\URCornerWallPaper{0.13}{img/dop/koptekstlogo.png}
\pagestyle{fancy}
\fancyhead{}
\renewcommand{\headrulewidth}{0pt}


\fancyfoot[L]{Release notes}
\fancyfoot[C]{  }
\fancyfoot[R]{\textbf{\thepage}\ / \pageref{LastPage} \linebreak \linebreak \_\_\_\_\_\_\_ }

% Settings for table of contents.	
\setcounter{secnumdepth}{4}
\setcounter{tocdepth}{3}


\makeatletter
% some extra spacing for the table of contents
\renewcommand{\l@subsection}{\@dottedtocline{2}{1.5em}{3em}}
\renewcommand{\l@subsubsection}{\@dottedtocline{2}{2.7em}{4em}}

\renewcommand\paragraph{%
   \@startsection{paragraph}{4}{0mm}%
      {-\baselineskip}%
      {.5\baselineskip}%
      {\normalfont\normalsize\bfseries}}
\makeatother

% Define variables
\newcommand{\customer}{Dimpact release 1.14.1}
\newcommand{\projectname}{Release notes}
\newcommand{\customerdomain}{Het klantdomein}
\newcommand{\authors}{E Alvares}


% Voorblad of the document
\ThisLRCornerWallPaper{0.8}{img/dop/voorbladlogo.png}

\title{\textbf{\customer} \\ \projectname}
\pretitle{\begin{flushleft}\LARGE}
\posttitle{\par\end{flushleft}}


\author{}  % skippen we voor maketitle
\date{}

% The actual Document:
\begin{document}
\maketitle
\vspace{-2.6cm}
\begin{flushright}
\begin{tabularx}{4.8cm}{ X }
Dutch Open Projects			\\
Doornseweg 12					\\	
3832 RL Leusden					\\
T: +31[0]33 - 4 50 50 50		\\
F: +31[0]33 - 4 50 50 57		
\\*
\\*
\\*
\\*
\\*
\\*
\\*
\\*
\\*
\\*
\\*
\\*
\\*
\\*
\\*
\\*
\\*
\\*
\footnotesize
\copyright All rights reserved.\\*
\footnotesize
No part of the contents of this publication may be reproduced, stored in a data processing system or transmitted in any form or by any means without the written permission of Dutch Open Projects B.V.
\end{tabularx}
\end{flushright}
  
 \null
 \vfill    
  \begin{tabularx}{\linewidth}{ p{4cm} X }
    Plaats & Leusden								\\
    Laatst bijgewerkt & \ddmmyyyydate \today		\\
    Auteurs & \authors							\\
    Versie & concept								\\
  \end{tabularx}
\pagebreak


% table of contents

\clearpage

%\renewcommand*\contentsname{Inhoudsopgave}
%\tableofcontents
%\pagebreak

\section{Wijzigingen}
\subsection{Functionele wijzigingen}
Onderstaande tabel geeft aan op welke punten het moedersjabloon functioneel is gewijzigd. Gebruikers van het systeem moeten er rekening mee houden dat het product op deze punten anders werkt dan voorheen.

\begin{tabular}{| r | p{15cm} |}
	\hline \# & Omschrijving \\ \hline \hline
	700 & Alfabet Menu aan en uit kunnen zetten bij voorkeur in thema per intranet/internet (Per abuis in eerste instantie niet vermeld bij 1.14) \\ \hline
\end{tabular}

\subsection{Opgeloste problemen}
Onderstaande tabel geeft aan welke problemen (\emph{bugs}) er binnen de geleverde versie zijn opgelost.

\begin{tabular}{| r | p{15cm} |}
	\hline \# & Omschrijving \\ \hline \hline

	727 & Carrousel niet zichtbaar bij eerste opstart \\ \hline
\end{tabular}

\subsection{Overige wijzigingen}

Onderstaande wijzigingen zijn doorgevoerd ten behoeve van webrichtlijnen. Een aantal van deze wijzigingen heeft een beperkte functionele impact.

\begin{tabular}{| r | p{15cm} |}
	\hline \# & Omschrijving \\ \hline \hline
	N.V.T. & N.V.T. \\ \hline
\end{tabular}

Onderstaande wijzigingen zijn gerealiseerd ten behoeve van de exploitatie van de Dimpact-websites, en hebben geen invloed op eindgebruikers.

\begin{tabular}{| r | p{15cm} |}
	\hline \# & Omschrijving \\ \hline \hline
	725 & Release 1.14.1 / Security updates \\ \hline
\end{tabular}

\section{Testresultaten}
De geleverde versie bevat geen \emph{known issues} anders dan de situatie rondom 727/721.

\end{document}
